\documentclass[a4paper,12 pt]{article}

%% Language and font encodings
\usepackage[spanish,es-tabla]{babel}
\usepackage[utf8]{inputenc}
\usepackage[T1]{fontenc}
\usepackage{cite}
\usepackage{amsmath, amsthm, amssymb, amsfonts} 
\usepackage[mathscr]{eucal}
\usepackage{ulem} % subrayar, tachar
\usepackage{hyperref} % para el url
\usepackage{multirow, array} % para las tablas
\usepackage{float} % para tablas [H]
\usepackage{graphicx} % graficos
\usepackage{titlesec} %subsubsubsection
\usepackage[usenames]{color} %palabras con color
\usepackage{lscape} % texto horizontal
\usepackage{pdflscape} % hoja horizontal
\usepackage{booktabs} %tabla con puntos

\newcolumntype{P}[1]{>{\centering\arraybackslash}p{#1}}
\newcolumntype{M}[1]{>{\centering\arraybackslash}m{#1}}
\newcolumntype{L}[1]{>{\raggedleft\arraybackslash}p{#1}}
\newcolumntype{R}[1]{>{\raggedright\arraybackslash}p{#1}}
  
%counter
\newcounter{mycounter} % create a new counter, called 'mycounter'
% default def'n of '\themycounter' is '\arabic{mycounter}'

%% command to increment 'mycounter' by 1 and to display its value:
\newcommand\showmycounter{\stepcounter{mycounter}\themycounter}

\usepackage{lipsum}
\newcommand\showlips{\stepcounter{mycounter}\lipsum[\value{mycounter}]}


%% Sets page size and margins
\usepackage[a4paper,top=2.5cm,bottom=2.5cm,left=2.5cm,right=2.5cm,marginparwidth= 1.5cm]{geometry}
  
\title{\LARGE \textbf{\\[0.5cm] Ingenio\\[2.5cm]}}

% Chicos, porfavor incluyan sus nombres en orden alfabetico
\author{\textbf{Grupo 7}\\[0.5cm]
        Valeria Huepa Ducuara\\
        Juan José Peña Becerra\\
        Carlos Daniel Rincón Mora\\
        Guiselle Tatiana Zambrano Penagos\\[2.5cm]
        Sebastian David Moreno Bernal\\[2.5cm]}
\date{}

\begin{document}

\newpage

\begin{figure}
    \centering
    \includegraphics[width=0.57\textwidth]{images/escudoUN.png}
\end{figure}
\maketitle 
\thispagestyle{empty}

\begin{center}
    \small Universidad Nacional de Colombia\\
    Facultad de Ingeniería\\
    Departamento de Ingeniería de Sistemas e Industrial\\
    Ingeniería de Sistemas\\
    Ingeniería de Software II\\
    Bogotá, Colombia\\
    2020
\end{center}
\newpage
\tableofcontents % indice de contenidos}
\thispagestyle{empty}

\newpage
\setcounter{page}{1}
\pagestyle{plain}

\section{Descripción General}


% arreglar: Mejorar la descripción del problema
\subsection{Descripción del Problema}

Dada la existencia de una necesidad de tener una plataforma online, adicionalmente
no existen muchos canales de información para personas interesadas en el campo de
la Ingeniería que quieran mantenerse informadas diariamente por medio de informes,
artículos y noticias de parte de una fuente confiable y en constante
actualización, esto es porque los canales existentes no se mantienen en continuo
mantenimiento o no son de agrado para el público en mención. Así mismo, hoy en día
circulan noticias falsas por redes sociales, y en otros medios de comunicación
quitando la certeza y veracidad de la información. Se desea una plataforma más
personal, en donde las personas del campo de la Ingeniería pueden conocerse e
interactuar entre ellas, para obtener conocimiento en diversas áreas en común de
interés para cada uno, además de comentar a los demás usuario su opinión de cierto
tema.

% arreglar: Mejorar la descripción del producto
\subsection{Descripción del Producto}

El software a realizar pretende tener un vínculo con el público interesado en
temas actuales de Ingeniería (Stakeholders), y desea brindarle información de
primera mano, así como artículos (científicos, de opinión y periodísticos ) de
gran interés para el mundo actual, con autores de renombre en la industria de
Ingeniería. Así mismo, podrán registrarse para recibir notificaciones cada que
algún artículo de su posible interés sea publicado. Recibirán notificaciones de
sus autores preferidos, para mantenerse actualizados con el trabajo de reconocidos
ingenieros.\\

Se desea realizar una plataforma más personal, que las que existen actualmente, en
la cual los usuarios puedan conocer a otras personas con intereses similares en el
campo de la Ingeniería, además de conocer, seguir y  comentar el trabajo de otros
ingenieros.\\

Foros, notificaciones..

La metodología a trabajar es Scrum.El manejo del código de software se maneja por
medio de un sistema de versionamiento central (VCS) tipo Git, usando la
herramienta GitHub. 

Fechas de las avance del proyecto será de forma semanal y los reportes de avances
internos, reuniones, serán de tipo diario.

% arreglar: Mejorar la descripción de los roles
\subsection{Descripción de los Roles de Usuario}

\begin{itemize}
    \item \textbf{Administrador} 
    \item \textbf{Autor} 
    \item \textbf{Usuario} 
    \item \textbf{Visitante} 
\end{itemize}{}

\subsection{Restricciones Generales}
\subsection{Supuestos y Dependencias}

\section{Descripción del Equipo de Desarrollo}

\subsection{Roles del equipo}

\begin{table}[H]
    \centering
    \small{
    \begin{tabular}{|c|c|}
        \hline
        \textbf{Rol}   &   \textbf{Nombre}  \\ 
        \hline
        Scrum Master   &   Guiselle Tatiana Zambrano Penagos\\
        \hline
        \multirow{2}{2cm}{Front End}    &   Valeria Huepa Ducuara\\
            &   Carlos Daniel Rincón Mora\\
        \hline
        \multirow{2}{2cm}{Back End}    &   Juan José Peña Becerra\\
            &   Guiselle Tatiana Zambrano Penagos\\
        \hline
         
    \end{tabular}
    \caption{Roles del equipo}}
    \label{T00}
\end{table}{}

\subsection{Recursos del equipo}

\begin{table}[H]
    \centering
    \small{
    \begin{tabular}{|M{2cm}|M{1.5cm}|M{3cm}|M{3cm}|M{1.5cm}|M{2cm}|}
        \hline
        \textbf{Nombre}    &\textbf{Marca}     &\textbf{Sistema Operativo} 
        &\textbf{Procesador}   &\textbf{RAM}   &\textbf{Capacidad}\\
        \hline
        Valeria Huepa                       &Asus   &Windows 10 Home x64 
        & Intel(R) Core(TM) i5-4210U CPU    &12GB   &HDD 240GB \\
        \hline
        Juan Peña                           &Asus   &Windows 10 Home x64
        &Intel(R) Core(TM) i5-6198DU CPU @ 2.30 GHz &8GB    &HDD 240GB\\
        \hline
        Carlos Rincón                       &Asus   &Windows 10 Home x64
        &Intel(R) Core(TM) i7-3537U CPU @ 2.00GHz   &8GB    &SSD 128GB\\
        \hline
        Tatiana Zambrano                    &Lenovo &Kali Linux x64
        &AMD A8 - 7410 2.2 GHz              &4GB    &HDD 500GB\\
        \hline
    \end{tabular}
    \caption{Recursos del equipo}
    \label{T01}}
\end{table}{}

% arreglar: Completar info
\subsection{Disposición del equipo}

\section{Historias de Usuario}

% plantilla - Copear y pegar
\begin{table}[H]
    \centering
    \small{
    \begin{tabular}{|M{3cm}|M{2cm}|M{3cm}|M{2cm}|}
        \hline
        \multicolumn{4}{|c|}{\textbf{Nombre}}\\
        \hline
        \textbf{ID} &\showmycounter &\textbf{Rol}                       &   \\
        \hline
        \textbf{Prioridad}          &   &\textbf{Riesgo en Desarrollo}  &   \\
        \hline
        \textbf{Puntos Esfuerzo}    &   &\textbf{Iteración Asignada}    &   \\
        \hline
        \multirow{2}{3cm}{\centering\textbf{Programadores asignados}} 
            &\multicolumn{3}{|c|}{ Pepito Perez }\\
            &\multicolumn{3}{|c|}{ Juanita Alcachofa}\\
        \hline
        \multirow{3}{3cm}{\centering\textbf{Descripción}}
            &\multicolumn{3}{|l|}{
                \textcolor{red}{Como} 
            }\\
            &\multicolumn{3}{|l|}{
                \textcolor{red}{Quiero} 
            }\\
            &\multicolumn{3}{|l|}{
                \textcolor{red}{Con el fin de} 
            }\\
        \hline
        \multirow{1}{3cm}{\centering\textbf{Validación}}
            &\multicolumn{3}{|l|}{
                
            }\\
        \hline
    \end{tabular}
    \caption{Nombre de la historia}
    \label{T02}}
\end{table}{}


\section{Mockups}
\begin{figure}[H]
    \centering
    \includegraphics[scale = 0.2]{images/escudoUN.png}
    \caption{Descripción}
    \label{F00}
\end{figure}{}


\begin{landscape}
\section{Modelo Entidad Relación}
    \begin{figure}[H]
        \centering
        \includegraphics[scale = 0.5]{images/escudoUN.png}
        \caption{Descripción}
        \label{F01}
    \end{figure}{}
\end{landscape}


\section{Modelo de Base de Datos}

\section{Estimación de Costos}

% plantilla: copiar y pegar
\begin{table}[H]
    \centering
    \small{
    \begin{tabular}{R{6cm}L{6cm}}
        \textbf{Concepto}   &\textbf{Valor}\\
        \\
        \multicolumn{2}{c}{Holi \dotfill \$10'000.000} \\
        \multicolumn{2}{c}{Holi \dotfill \$10'000.000} \\
        \hline
        \multicolumn{2}{c}{\textbf{Total} \dotfill \$10'000.000} \\
    \end{tabular}
    \label{T03}}
\end{table}{}


\section{Análisis de riesgos}

\begin{thebibliography}{50}
\bibitem{00} Description \url{https://github.com}

\end{thebibliography}{}

\end{document}{}
